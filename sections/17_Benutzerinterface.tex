\section{Benutzerinterface}

\subsection{Layout}
Das Layout beruht auf der gegebenen Blynk Oberfläche und einer Auswahl an verfügbaren Widgets. Wir mussten dabei auf eine Legacy Version der App zurückgreifen, da die möglichkeit zur Verbindung mit Bluetooth eingestellt wurde.

\begin{figure}[H]
    \begin{center}
    \includegraphics[width=5cm]{Layout_BlynkApp.jpg}
    \end{center}
    \caption{Layout in der Blynk App}
\end{figure}

Es soll mindestens ein Killswitch, eine Bluetooth-Verbindung, ein GPS-Stream und ein Terminal darin enthalten sein.

\subsection{Funktionen}
Über den Killswitch soll der Roboter in den «Fahren»-Modus versetzt werden können und diesen wieder verlassen. Das Bluetooth- und GPS-Widget dienen lediglich dem Informationsaustausch zwischen dem Ardu-ino und der App. Der Terminal soll schliesslich die Möglichkeit bieten, GPS-Koordinaten direkt dem Arduino zuzuführen, zu welchen er fahren soll.
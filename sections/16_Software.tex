\section{Software}

\subsection{Softwareverweise}
Als Grundlage haben wir uns für die Software des Projekts von Hackster.io bedient. Es handelt sich hierbei um 2 Herren, die dasselbe Projekt bereits realisiert hatten und eine Dokumentation dazu online zur Verfügung gestellt haben. Darunter auch ihre Software.\\
\\
Die Idee war, das Projekt mit dieser Vorlage zum Laufen zu bringen, und von dort aus Verbesserungen vorzunehmen. \\
\\
Nachdem Anfangs die Bibliotheken der Arduino IDE für den Kompass und das GPS-Modul ausgetauscht werden mussten, stellte uns die Blynk-App, resp. Blynk Plattform vor das Nächste Problem, indem sie die Möglichkeit der Bluetooth Konnektivität seit Mai 2022 aus dem Sortiment genommen haben. Wir mussten also eine Legacy-Version der App suchen, welche funktionierte. Das klappte auch, bis wir kurz vor dem Jahreswechsel freundlicherweise von der Blynk-App darauf hingewiesen wurden, dass die Legacy-Plattform, welche für das Login in der App benötigt wird per 01.01.2023 vom Netz genommen wird.\\
\\
Nach anfänglicher Panik, haben wir es jedoch geschafft, auf Hetzner.de einen eigenen Legacy-Server einzurichten, wodurch wir unseren Stand zu diesem Zeitpunkt nicht verwerfen mussten. 
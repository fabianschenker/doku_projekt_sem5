\section{Zielformulierung}

\subsection{Ziele und Nutzen des Auftraggebers}
Der Auftraggeber wünscht sich ein lauffähiges Produkt, welches alle typischen Eigenschaften eines mechatronischen Systems enthält. Dabei sollen die Kenntnisse aus den Semestern des Studiums angewendet werden.

\subsection{Ziele und Nutzen des Anwenders}
Der Anwender erwartet vom System eine simple Bedienung und eine unkomplizierte Benutzer-Erfahrung. Er möchte möglichst wenig Schritte ausführen, um den Roboter zum Folgen zu bewegen. Wo immer ein Mensch mit Badelatschen geht, ausser ins Wasser natürlich, sollte der Roboter folgen können.

\subsection{Anforderungen}
Anforderungen an das System sind, dass sich eine Kühlbox auf Rädern selbst fortbewegen kann, wobei es leichtes Gelände bewältigen können muss, wie zum Beispiel eine Wiese. Des Weiteren sollen die darin aufbewahrten Getränke deutlich unter Lufttemperatur temperiert werden.

\subsection{Zielkatalog}
\begin{table}[H]
    \centering
    \caption{Zielkatalog}
    \label{tab:my-table}
    \resizebox{\columnwidth}{!}{%
    \begin{tabular}{|llllll|}
    \hline
    \multicolumn{1}{|l|}{\textbf{Objekt}} & \multicolumn{1}{l|}{\textbf{Eigenschaft}}                                                        & \multicolumn{1}{l|}{\textbf{Ausmass}} & \multicolumn{1}{l|}{\textbf{Zeitpunkt}} & \multicolumn{1}{l|}{\textbf{Zielart}} & \textbf{Priorität} \\ \hline
    \multicolumn{6}{|l|}{\textbf{Level 1}}                                                                                                                                                                                                                                                  \\ \hline
    \multicolumn{1}{|l|}{Gerüst}          & \multicolumn{1}{l|}{Aufbau des Roboters   mit Fahrwerk, Halterung für Kühlbox und Elektronik}    & \multicolumn{1}{l|}{zusammengebaut}   & \multicolumn{1}{l|}{KW 46}              & \multicolumn{1}{l|}{M}                & -                  \\ \hline
    \multicolumn{1}{|l|}{Follow-Funktion} & \multicolumn{1}{l|}{Roboter kann   autonom jemandem nachfahren}                                  & \multicolumn{1}{l|}{erreicht}         & \multicolumn{1}{l|}{KW 50}              & \multicolumn{1}{l|}{M}                & -                  \\ \hline
    \multicolumn{1}{|l|}{Isolierte Box}   & \multicolumn{1}{l|}{Eine Kiste, welche   die Innentemperatur von der Aussentemperatur isoliert.} & \multicolumn{1}{l|}{}                 & \multicolumn{1}{l|}{KW 44}              & \multicolumn{1}{l|}{M}                & -                  \\ \hline
    \multicolumn{1}{|l|}{Kühlen}          & \multicolumn{1}{l|}{Die isolierte Box   soll gekühlt werden}                                     & \multicolumn{1}{l|}{< 5 °C}           & \multicolumn{1}{l|}{KW 46}              & \multicolumn{1}{l|}{R}                & 100                \\ \hline
    \multicolumn{1}{|l|}{Akku}            & \multicolumn{1}{l|}{Der Akku soll für   3h aktive Kühlung und 1h Fahren ausreichen}              & \multicolumn{1}{l|}{erreicht}         & \multicolumn{1}{l|}{KW 48}              & \multicolumn{1}{l|}{M}                & -                  \\ \hline
    \multicolumn{1}{|l|}{Kapazität}       & \multicolumn{1}{l|}{Es soll mindestens   Platz für 12 0.5 L Dosen haben}                         & \multicolumn{1}{l|}{erreicht}         & \multicolumn{1}{l|}{KW 44}              & \multicolumn{1}{l|}{M}                & -                  \\ \hline
    \multicolumn{6}{|l|}{\textbf{Level 2}}                                                                                                                                                                                                                                                  \\ \hline
    \multicolumn{1}{|l|}{Wechsel Akku}    & \multicolumn{1}{l|}{Akku soll   auswechselbar sein}                                              & \multicolumn{1}{l|}{erreicht}         & \multicolumn{1}{l|}{KW 48}              & \multicolumn{1}{l|}{W}                & 40                 \\ \hline
    \multicolumn{1}{|l|}{Deckel}          & \multicolumn{1}{l|}{Der Deckel soll sich   automatisch öffnen können}                            & \multicolumn{1}{l|}{erreicht}         & \multicolumn{1}{l|}{KW 49}              & \multicolumn{1}{l|}{W}                & 30                 \\ \hline
    \multicolumn{1}{|l|}{All-Terrain}     & \multicolumn{1}{l|}{Der Roboter soll   auch über kleinere Hindernisse fahren können}             & \multicolumn{1}{l|}{5cm Schwelle}     & \multicolumn{1}{l|}{KW 49}              & \multicolumn{1}{l|}{O}                & 30                 \\ \hline
    \multicolumn{1}{|l|}{Federung}        & \multicolumn{1}{l|}{Einbau einer   Federung/Dämpfung}                                            & \multicolumn{1}{l|}{erreicht}         & \multicolumn{1}{l|}{}                   & \multicolumn{1}{l|}{W}                & 20                 \\ \hline
    \multicolumn{6}{|l|}{\textbf{Level 3}}                                                                                                                                                                                                                                                  \\ \hline
    \multicolumn{1}{|l|}{Leine}           & \multicolumn{1}{l|}{Leine, an der der   Roboter gezogen werden kann, falls der Akku leer ist.}   & \multicolumn{1}{l|}{erreicht}         & \multicolumn{1}{l|}{}                   & \multicolumn{1}{l|}{W}                & 30                 \\ \hline
    \multicolumn{1}{|l|}{Variable Temp.}  & \multicolumn{1}{l|}{Die Temperatur in   der Kühlbox soll variabel geregelt werden}               & \multicolumn{1}{l|}{erreicht}         & \multicolumn{1}{l|}{}                   & \multicolumn{1}{l|}{W}                & 10                 \\ \hline
    \multicolumn{1}{|l|}{Wärmen}          & \multicolumn{1}{l|}{Die Isolierte Box   soll heizbar sein}                                       & \multicolumn{1}{l|}{Warm 50° C}       & \multicolumn{1}{l|}{}                   & \multicolumn{1}{l|}{W}                & 40                 \\ \hline
    \end{tabular}%
    }
    \end{table}

\subsection{Benutzer / Zielgruppe}
\begin{table}[H]
    \centering
    \caption{Benutzer / Zielgruppe}
    \label{tab:my-table}
    \begin{tabular}{|l|l|l|}
    \hline
    \rowcolor[HTML]{E0E0E0} 
    \textbf{Zielgruppe} & \textbf{Name}                & \textbf{Beschreibung}            \\ \hline
    Stakeholder         & FHNW, Silvan Wirth           & Ansprechperson / Auftraggeber    \\ \hline
    Anwender            & Prof. Dr. Robert   Alard     & Kunde                            \\ \hline
    Zielgruppe          & Studenten/Familien/Marketing & Potentielle Studieninteressierte \\ \hline
    \end{tabular}
    \end{table}
\section*{Rahmenbedingungen}
\addcontentsline{toc}{section}{Rahmenbedingungen}

\subsection*{Übersicht}
\begin{itemize}
    \item Zu realisierende Projekte können vorgeschlagen werden (Berücksichtigung gewisser Rahmenbedingungen)
    \item Bearbeitung in Zweier-oder Dreiergruppen
    \item Vorgabe SGL: lauffähige Produkte mit Aussenwirkung für Marketing
    \item Zuweisung von max. zwei Gruppen auf ein Projekt
    \item Austausch zwischen Gruppen möglich (Lösungsvarianz höher, Chance, dass ein lauffähiges Produkt entsteht, ist höher)
\end{itemize}
 

\subsection*{Rahmenbedingungen Projekte}
\begin{itemize}
    \item Enthält Merkmale eines mechatronischen Systems (Sensorik, Aktorik, Informationsverarbeitung (Rechner / Steuerung))
    \item Aktorik: physische Bewegung muss vorhanden sein
    \item User-Interface: optional, nicht zwingend
    \item Energieversorgung sinnvoll gelöst    
\end{itemize}

\subsection*{Qualifikationsziele und Kompetenzen}
\begin{itemize}
    \item Sachkompetenz:
    \begin{itemize}
        \item Verstehen und praktisch Anwenden der mechatronischen Modellbildung
    \end{itemize}
    \item Selbstkompetenz:
    \item \begin{itemize}
        \item Eigenständige Auswahl und Einsatz von Aktoren, Sensoren, Mikrorechner
    \end{itemize}
    \item Sozial-ethische Kompetenz:
    \begin{itemize}
        \item ein System vom Konzept bis zum funktionierenden Produkt entwickeln.
        \item mechatronisches Projekt im Team erfolgreich planen und durchführen.
        \item Gruppendynamische Prozesse bei der Bearbeitung größerer Aufgaben innerhalb von Projektgruppen erfahren
    \end{itemize}
\end{itemize}

\subsection*{Unterstützung des Lernprozesses}
\begin{itemize}
    \item Aufwand gemäss Modulhandbuch: 15 h Präsenz, 45 h Selbststudium
    \item Anwenden des theoretischen Wissens aus der Vorlesung mechatronische Systeme (und aller anderer vorgängiger Vorlesungen)
    \item Praktische Realisierung wird begleitet durch Dozenten.    
\end{itemize}

\subsection*{Verknüpfung in Modulgruppe Mechatronik III}
\begin{itemize}
    \item Fach «Mechatronische Systeme» wirkt auf Fach «Mechatronisches Labor»
    \item Inhalte und Methoden aus Vorlesung sollen angewendet werden
    \item Unterstützung bei Realisierung eines konkreten Projektes
\end{itemize}

\subsection*{Prüfungsbedingungen}
\begin{itemize}
    \item Leistungserfassung durch Präsentation und Bewertung des Projektes
    \item Termin: 10.01.2023	08:30 – 12:15
    \item Gewichtung: Schlussnote gemäss Bewertungsraster
\end{itemize}

\subsection*{Youtube-Videos / Mechatronik-Trinational Channel}
Zu den Projekten werden Videos gestaltet. Diese fliessen in die Bewertung der Projekte mit ein (siehe Bewertungsraster).\\
\\
Das Video Ihres Projektes muss «Youtube-konform» sein, z.B. sind die Musik-Copyrights zu beachten und wenn Sie –aus welchen Gründen auch immer-nicht im Video zu sehen sein wollen, dann sollten Sie sich auch nicht selbst aufnehmen.

\subsection*{Projektrahmen}
\begin{itemize}
    \item Kostendeckel CHF 200.-pro Projekt
    \item Kann ggf. bei marketingtechnischem Nutzen erhöht werden (nach vorheriger Absprache)
    \item Materialbeschaffung selber organisieren
    \item Abrechnung über Sekretariat/Studiengangsleitung Ende Semester
    \item Aufstellung mit Belegen und Kontoverbindung (IBAN) gemäss Formular
    \item muss vorgängig durch Dozenten genehmigt werden
    \item Arduino und/oder RaspberryPi wird abgegeben durch Labor (muss nicht in den Projektkosten berücksichtigt werden)
\end{itemize}

\subsection*{Laborzugang und Werkstattbenutzung}
\begin{itemize}
    \item Die Studierenden dürfen aus Sicherheits-und Versicherungsgründen das Labor Mechatronik Trinational / Campus Muttenz nicht unbegleitet (ohne Dozierenden) nutzen
    \item Zugang Labor Mechatronik Trinational gem. Unterrichtszeiten
    \item Werkstattaufträge (extern oder Hochschulen) sind vorgängig mit dem Dozierenden abzusprechen
    \item Kosten sind von ins Projektbudget 
\end{itemize}
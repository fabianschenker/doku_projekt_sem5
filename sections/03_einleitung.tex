\section{Einleitung}

\subsection{Sinn und Zweck der Dokumentation}
Diese Dokumentation des Projekts hält die einzelnen Phasen und Arbbeitsschritte fest. Es soll unsere Gedankengänge und Entscheidungen nachvollziehbar aufzeigen und begründen. Es beinhaltet die Anforderungen, welche an die Projektarbeit des 5. Semesters gestellt werden.

\subsection{Vision (Inhalt und Ziele)}
Der Following Beercooler soll den Transport von Bier zu einem Tag am Strand oder auf der Wiese etwas angenehmer gestalten, indem er das Schleppen der Getränke selbst übernimmt. So besteht die Möglichkeit alle restlichen Sachen einfacher tragen zu können und die schweren Bierdosen fahren mit den Beercooler ohne weiteren Aufwand hinter einem her. Am Zielort angekommen sorgt der Beercooler dafür, dass die Getränke noch deutlich länger als in einer normalen Kühlbox kalt bleiben. 
\\ \\
Die Basis kann indes auch für den Transport anderer Dinge benutzt werden, da die Kühlbox im finalen Design des Systems demontierbar ist.

\subsection{Definitionen und Abkürzungen}
\begin{table}[H]
    \label{tab:abkuerzungen}
    \begin{tabular}{lll}
    Bzw.  &  & Beziehungsweise                             \\
    Resp. &  & Respektive                                  \\
    MDF   &  & Mitteldichte Faserplatte                    \\
    GPS   &  & Global Position System                      \\
    SDA   &  & System Data                                 \\
    SCL   &  & System Clock                                \\
    PWM   &  & Pulsweitenmodulation                        \\
    UART  &  & Universal Asynchronous Receiver Transmitter \\
    I2C   &  & Interner Integrated Circuit Bus             \\
    Akku  &  & Akkumulator                                 \\
    DC    &  & Gleichstrom                                 \\
    LiPo  &  & Lithium Polymer                             \\
    MISO  &  & Master in Slave out                         \\
    MOSI  &  & Master out Slave in                         \\
    IDE   &  & Integrated development environment         
    \end{tabular}
\end{table}

\subsection{Ablage, Gültigkeit und Bezüge zu anderen Dokumenten}
Das Dokument bezieht sich auf die Vorlesung «Mechatronisches Labor» und deren Bewertungskriterien, welche für das Semesterprojekt der Mechatronik Trinational gelten. Alle verwendeten Quellen sind im Quellenverzeichnis/Literaturverzeichnis  angegeben.

\subsection{Verteiler und Freigaben}

\begin{table}[H]
    \centering
    \caption{Verteiler und Freigaben}
    \begin{tabular}{|l|l|l|} 
    \hline
    \rowcolor[rgb]{0.753,0.753,0.753} \textbf{ Rolle / Rollen }          & \textbf{ Name }                                                        & \textbf{ E-Mail }                 \\ 
    \hline
    \begin{tabular}[c]{@{}l@{}}Projektleiter \\Hardware\end{tabular}     & Gass Matthias                                                          & matthias.gass@students.fhnw.ch    \\ 
    \hline
    \begin{tabular}[c]{@{}l@{}}Projektleiter \\Software\end{tabular}     & Knauber Max                                                            & max.knauber@students.fhnw.ch      \\ 
    \hline
    \begin{tabular}[c]{@{}l@{}}Projektleiter \\Konstruktion\end{tabular} & Schenker Fabian                                                        & fabian.schenker@students.fhnw.ch  \\ 
    \hline
    Kunde                                                                & \begin{tabular}[c]{@{}l@{}}Vertreten durch \\Silvan Wirth\end{tabular} & silvan.wirth@fhnw.ch              \\ 
    \hline
    Anwender                                                             & Robert Alard                                                           & robert.alard@fhnw.ch              \\
    \hline
    \end{tabular}
    \end{table}
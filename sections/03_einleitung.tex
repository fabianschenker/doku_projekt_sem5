\section{Einleitung}

\subsection{Sinn und Zweck des Dokuments}
Das Dokument soll das Projekt dokumentieren und begleiten. Es soll unsere Gedankengänge und Entscheidungen nachvollziehbar aufzeigen und begründen. Es beinhaltet die Anforderungen, welche an die Projektarbeit des 5. Semesters gestellt werden.

\subsection{Vision (Inhalt und Ziele)}
Der Following Beercooler soll den Transport von Bier zu einem Tag am Strand oder auf der Wiese etwas angenehmer gestalten, indem er das Schleppen der Getränke selbst übernimmt. Während man also mit allem anderen beladen ist, muss man den Beercooler nicht noch zusätzlich tragen oder gar in einem zweiten Anlauf holen. Am Zielort angekommen sorgt der Beercooler dafür, dass die Getränke noch deutlich länger kalt bleiben als in einer normalen Kühlbox. \\
\\
Die Basis, falls die Kühlbox demontierbar wird, könnte indes auch für den Transport anderer Dinge benutzt werden.

\subsection{Definitionen und Abkürzungen}

\subsection{Ablage, Gültigkeit und Bezüge zu anderen Dokumenten}
Das Dokument bezieht sich auf die Vorlesung Mechatronisches Labor und deren Bewertungskriterien, welche für das Semesterprojekt der Mechatronik Trinational gelten. Alle verwendeten Quellen sind in den Quellenangaben angegeben.

\subsection{Verteiler und Freigaben}

\begin{table}[H]
    \centering
    \caption{Verteiler und Freigaben}
    \begin{tabular}{|l|l|l|} 
    \hline
    \rowcolor[rgb]{0.753,0.753,0.753} \textbf{ Rolle / Rollen }          & \textbf{ Name }                                                        & \textbf{ E-Mail }                 \\ 
    \hline
    \begin{tabular}[c]{@{}l@{}}Projektleiter \\Hardware\end{tabular}     & Gass Matthias                                                          & matthias.gass@students.fhnw.ch    \\ 
    \hline
    \begin{tabular}[c]{@{}l@{}}Projektleiter \\Software\end{tabular}     & Knauber Max                                                            & max.knauber@students.fhnw.ch      \\ 
    \hline
    \begin{tabular}[c]{@{}l@{}}Projektleiter \\Konstruktion\end{tabular} & Schenker Fabian                                                        & fabian.schenker@students.fhnw.ch  \\ 
    \hline
    Kunde                                                                & \begin{tabular}[c]{@{}l@{}}Vertreten durch \\Silvan Wirth\end{tabular} & silvan.wirth@fhnw.ch              \\ 
    \hline
    Anwender                                                             & Robert Alard                                                           & robert.alard@fhnw.ch              \\
    \hline
    \end{tabular}
    \end{table}